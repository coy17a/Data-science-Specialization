\documentclass[]{article}
\usepackage{lmodern}
\usepackage{amssymb,amsmath}
\usepackage{ifxetex,ifluatex}
\usepackage{fixltx2e} % provides \textsubscript
\ifnum 0\ifxetex 1\fi\ifluatex 1\fi=0 % if pdftex
  \usepackage[T1]{fontenc}
  \usepackage[utf8]{inputenc}
\else % if luatex or xelatex
  \ifxetex
    \usepackage{mathspec}
  \else
    \usepackage{fontspec}
  \fi
  \defaultfontfeatures{Ligatures=TeX,Scale=MatchLowercase}
\fi
% use upquote if available, for straight quotes in verbatim environments
\IfFileExists{upquote.sty}{\usepackage{upquote}}{}
% use microtype if available
\IfFileExists{microtype.sty}{%
\usepackage{microtype}
\UseMicrotypeSet[protrusion]{basicmath} % disable protrusion for tt fonts
}{}
\usepackage[margin=1in]{geometry}
\usepackage{hyperref}
\hypersetup{unicode=true,
            pdftitle={Peer-graded Assignment: Statistical Inference Course ProjectPart 1},
            pdfauthor={Alejandro Coy},
            pdfborder={0 0 0},
            breaklinks=true}
\urlstyle{same}  % don't use monospace font for urls
\usepackage{color}
\usepackage{fancyvrb}
\newcommand{\VerbBar}{|}
\newcommand{\VERB}{\Verb[commandchars=\\\{\}]}
\DefineVerbatimEnvironment{Highlighting}{Verbatim}{commandchars=\\\{\}}
% Add ',fontsize=\small' for more characters per line
\usepackage{framed}
\definecolor{shadecolor}{RGB}{248,248,248}
\newenvironment{Shaded}{\begin{snugshade}}{\end{snugshade}}
\newcommand{\KeywordTok}[1]{\textcolor[rgb]{0.13,0.29,0.53}{\textbf{#1}}}
\newcommand{\DataTypeTok}[1]{\textcolor[rgb]{0.13,0.29,0.53}{#1}}
\newcommand{\DecValTok}[1]{\textcolor[rgb]{0.00,0.00,0.81}{#1}}
\newcommand{\BaseNTok}[1]{\textcolor[rgb]{0.00,0.00,0.81}{#1}}
\newcommand{\FloatTok}[1]{\textcolor[rgb]{0.00,0.00,0.81}{#1}}
\newcommand{\ConstantTok}[1]{\textcolor[rgb]{0.00,0.00,0.00}{#1}}
\newcommand{\CharTok}[1]{\textcolor[rgb]{0.31,0.60,0.02}{#1}}
\newcommand{\SpecialCharTok}[1]{\textcolor[rgb]{0.00,0.00,0.00}{#1}}
\newcommand{\StringTok}[1]{\textcolor[rgb]{0.31,0.60,0.02}{#1}}
\newcommand{\VerbatimStringTok}[1]{\textcolor[rgb]{0.31,0.60,0.02}{#1}}
\newcommand{\SpecialStringTok}[1]{\textcolor[rgb]{0.31,0.60,0.02}{#1}}
\newcommand{\ImportTok}[1]{#1}
\newcommand{\CommentTok}[1]{\textcolor[rgb]{0.56,0.35,0.01}{\textit{#1}}}
\newcommand{\DocumentationTok}[1]{\textcolor[rgb]{0.56,0.35,0.01}{\textbf{\textit{#1}}}}
\newcommand{\AnnotationTok}[1]{\textcolor[rgb]{0.56,0.35,0.01}{\textbf{\textit{#1}}}}
\newcommand{\CommentVarTok}[1]{\textcolor[rgb]{0.56,0.35,0.01}{\textbf{\textit{#1}}}}
\newcommand{\OtherTok}[1]{\textcolor[rgb]{0.56,0.35,0.01}{#1}}
\newcommand{\FunctionTok}[1]{\textcolor[rgb]{0.00,0.00,0.00}{#1}}
\newcommand{\VariableTok}[1]{\textcolor[rgb]{0.00,0.00,0.00}{#1}}
\newcommand{\ControlFlowTok}[1]{\textcolor[rgb]{0.13,0.29,0.53}{\textbf{#1}}}
\newcommand{\OperatorTok}[1]{\textcolor[rgb]{0.81,0.36,0.00}{\textbf{#1}}}
\newcommand{\BuiltInTok}[1]{#1}
\newcommand{\ExtensionTok}[1]{#1}
\newcommand{\PreprocessorTok}[1]{\textcolor[rgb]{0.56,0.35,0.01}{\textit{#1}}}
\newcommand{\AttributeTok}[1]{\textcolor[rgb]{0.77,0.63,0.00}{#1}}
\newcommand{\RegionMarkerTok}[1]{#1}
\newcommand{\InformationTok}[1]{\textcolor[rgb]{0.56,0.35,0.01}{\textbf{\textit{#1}}}}
\newcommand{\WarningTok}[1]{\textcolor[rgb]{0.56,0.35,0.01}{\textbf{\textit{#1}}}}
\newcommand{\AlertTok}[1]{\textcolor[rgb]{0.94,0.16,0.16}{#1}}
\newcommand{\ErrorTok}[1]{\textcolor[rgb]{0.64,0.00,0.00}{\textbf{#1}}}
\newcommand{\NormalTok}[1]{#1}
\usepackage{graphicx,grffile}
\makeatletter
\def\maxwidth{\ifdim\Gin@nat@width>\linewidth\linewidth\else\Gin@nat@width\fi}
\def\maxheight{\ifdim\Gin@nat@height>\textheight\textheight\else\Gin@nat@height\fi}
\makeatother
% Scale images if necessary, so that they will not overflow the page
% margins by default, and it is still possible to overwrite the defaults
% using explicit options in \includegraphics[width, height, ...]{}
\setkeys{Gin}{width=\maxwidth,height=\maxheight,keepaspectratio}
\IfFileExists{parskip.sty}{%
\usepackage{parskip}
}{% else
\setlength{\parindent}{0pt}
\setlength{\parskip}{6pt plus 2pt minus 1pt}
}
\setlength{\emergencystretch}{3em}  % prevent overfull lines
\providecommand{\tightlist}{%
  \setlength{\itemsep}{0pt}\setlength{\parskip}{0pt}}
\setcounter{secnumdepth}{0}
% Redefines (sub)paragraphs to behave more like sections
\ifx\paragraph\undefined\else
\let\oldparagraph\paragraph
\renewcommand{\paragraph}[1]{\oldparagraph{#1}\mbox{}}
\fi
\ifx\subparagraph\undefined\else
\let\oldsubparagraph\subparagraph
\renewcommand{\subparagraph}[1]{\oldsubparagraph{#1}\mbox{}}
\fi

%%% Use protect on footnotes to avoid problems with footnotes in titles
\let\rmarkdownfootnote\footnote%
\def\footnote{\protect\rmarkdownfootnote}

%%% Change title format to be more compact
\usepackage{titling}

% Create subtitle command for use in maketitle
\newcommand{\subtitle}[1]{
  \posttitle{
    \begin{center}\large#1\end{center}
    }
}

\setlength{\droptitle}{-2em}

  \title{Peer-graded Assignment: Statistical Inference Course ProjectPart 1}
    \pretitle{\vspace{\droptitle}\centering\huge}
  \posttitle{\par}
    \author{Alejandro Coy}
    \preauthor{\centering\large\emph}
  \postauthor{\par}
      \predate{\centering\large\emph}
  \postdate{\par}
    \date{2019-02-15}


\begin{document}
\maketitle

\section{Peer-graded Assignment: Statistical Inference Course
Project}\label{peer-graded-assignment-statistical-inference-course-project}

The project consists of two parts:

\begin{enumerate}
\def\labelenumi{\arabic{enumi}.}
\tightlist
\item
  A simulation exercise.
\item
  Basic inferential data analysis.
\end{enumerate}

\subsection{Part 1: Simulation Excercise
:}\label{part-1-simulation-excercise}

In this project you will investigate the exponential distribution in R
and compare it with the Central Limit Theorem. The exponential
distribution can be simulated in R with rexp(n, lambda) where lambda is
the rate parameter. The mean of exponential distribution is 1/lambda and
the standard deviation is also 1/lambda. Set lambda = 0.2 for all of the
simulations. You will investigate the distribution of averages of 40
exponentials. Note that you will need to do a thousand simulations.

\subsubsection{Show the sample mean and compare it to the theoretical
mean of the
distribution.}\label{show-the-sample-mean-and-compare-it-to-the-theoretical-mean-of-the-distribution.}

Theoretical Mean of distribution is 1/ lambda if lmabda is 0.2, the
theoretical mean is 5

\begin{Shaded}
\begin{Highlighting}[]
\KeywordTok{set.seed}\NormalTok{(}\DecValTok{150}\NormalTok{)}
\NormalTok{lambda =}\StringTok{ }\FloatTok{0.2}
\NormalTok{n =}\StringTok{ }\DecValTok{40} \CommentTok{#number of exponentials}
\NormalTok{n_simulations =}\DecValTok{1000}

\NormalTok{exp_mean =}\StringTok{ }\OtherTok{NULL}
\ControlFlowTok{for}\NormalTok{(i }\ControlFlowTok{in} \DecValTok{1}\OperatorTok{:}\NormalTok{n_simulations)\{}
\NormalTok{  exp_mean =}\StringTok{ }\KeywordTok{c}\NormalTok{(exp_mean,}\KeywordTok{mean}\NormalTok{(}\KeywordTok{rexp}\NormalTok{(n,lambda)))}
\NormalTok{\}}
\KeywordTok{mean}\NormalTok{(exp_mean)}
\end{Highlighting}
\end{Shaded}

\begin{verbatim}
## [1] 5.00215
\end{verbatim}

Result indicate that the mean of 40 exponentials for 10000 simulations
is close to the theoretical mean of 5.

\subsubsection{Show how variable the sample is (via variance) and
compare it to the theoretical variance of the
distribution.}\label{show-how-variable-the-sample-is-via-variance-and-compare-it-to-the-theoretical-variance-of-the-distribution.}

The theoretical standart deviation is 1/lambda/sqrt(n) if, lamda is 0.2,
the standart deviations is 5 and theoretical variance is 0.79

\begin{Shaded}
\begin{Highlighting}[]
\NormalTok{theo_variance <-}\StringTok{ }\DecValTok{1}\OperatorTok{/}\NormalTok{lambda}\OperatorTok{/}\KeywordTok{sqrt}\NormalTok{(n)}
\NormalTok{variance_exp <-}\StringTok{ }\KeywordTok{var}\NormalTok{(exp_mean)}
\NormalTok{variance_exp}
\end{Highlighting}
\end{Shaded}

\begin{verbatim}
## [1] 0.6498806
\end{verbatim}

\begin{Shaded}
\begin{Highlighting}[]
\NormalTok{theo_variance}
\end{Highlighting}
\end{Shaded}

\begin{verbatim}
## [1] 0.7905694
\end{verbatim}

In this case the simulation is a little further.

\subsubsection{Show that the distribution is approximately
normal.}\label{show-that-the-distribution-is-approximately-normal.}

To probe normality the exp\_mean could be plotted in a histrogram

\begin{Shaded}
\begin{Highlighting}[]
\KeywordTok{library}\NormalTok{(ggplot2)}

\NormalTok{data <-}\StringTok{ }\KeywordTok{data.frame}\NormalTok{(exp_mean,n)}
\KeywordTok{ggplot}\NormalTok{(}\DataTypeTok{data =}\NormalTok{ data,}\KeywordTok{aes}\NormalTok{(exp_mean))}\OperatorTok{+}
\StringTok{  }\KeywordTok{geom_histogram}\NormalTok{(}\KeywordTok{aes}\NormalTok{(}\DataTypeTok{y=}\NormalTok{..density..),}\DataTypeTok{fill =} \StringTok{"steelBlue"}\NormalTok{,}\DataTypeTok{binwidth =} \FloatTok{0.2}\NormalTok{)}\OperatorTok{+}
\StringTok{  }\KeywordTok{stat_function}\NormalTok{(}\DataTypeTok{fun =}\NormalTok{ dnorm,}\DataTypeTok{args=} \KeywordTok{list}\NormalTok{(}\DataTypeTok{mean=}\DecValTok{1}\OperatorTok{/}\NormalTok{lambda,}\DataTypeTok{sd=}\DecValTok{1}\OperatorTok{/}\NormalTok{lambda}\OperatorTok{/}\KeywordTok{sqrt}\NormalTok{(}\DecValTok{40}\NormalTok{)),}\DataTypeTok{color =}\StringTok{"red"}\NormalTok{)}
\end{Highlighting}
\end{Shaded}

\includegraphics{Inferential_Statistics_Project_files/figure-latex/unnamed-chunk-3-1.pdf}


\end{document}
