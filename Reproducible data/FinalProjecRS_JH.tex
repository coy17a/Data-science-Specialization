\documentclass[]{article}
\usepackage{lmodern}
\usepackage{amssymb,amsmath}
\usepackage{ifxetex,ifluatex}
\usepackage{fixltx2e} % provides \textsubscript
\ifnum 0\ifxetex 1\fi\ifluatex 1\fi=0 % if pdftex
  \usepackage[T1]{fontenc}
  \usepackage[utf8]{inputenc}
\else % if luatex or xelatex
  \ifxetex
    \usepackage{mathspec}
  \else
    \usepackage{fontspec}
  \fi
  \defaultfontfeatures{Ligatures=TeX,Scale=MatchLowercase}
\fi
% use upquote if available, for straight quotes in verbatim environments
\IfFileExists{upquote.sty}{\usepackage{upquote}}{}
% use microtype if available
\IfFileExists{microtype.sty}{%
\usepackage{microtype}
\UseMicrotypeSet[protrusion]{basicmath} % disable protrusion for tt fonts
}{}
\usepackage[margin=1in]{geometry}
\usepackage{hyperref}
\hypersetup{unicode=true,
            pdftitle={FinalProjectRS JH},
            pdfauthor={Alejandro Coy},
            pdfborder={0 0 0},
            breaklinks=true}
\urlstyle{same}  % don't use monospace font for urls
\usepackage{color}
\usepackage{fancyvrb}
\newcommand{\VerbBar}{|}
\newcommand{\VERB}{\Verb[commandchars=\\\{\}]}
\DefineVerbatimEnvironment{Highlighting}{Verbatim}{commandchars=\\\{\}}
% Add ',fontsize=\small' for more characters per line
\usepackage{framed}
\definecolor{shadecolor}{RGB}{248,248,248}
\newenvironment{Shaded}{\begin{snugshade}}{\end{snugshade}}
\newcommand{\KeywordTok}[1]{\textcolor[rgb]{0.13,0.29,0.53}{\textbf{#1}}}
\newcommand{\DataTypeTok}[1]{\textcolor[rgb]{0.13,0.29,0.53}{#1}}
\newcommand{\DecValTok}[1]{\textcolor[rgb]{0.00,0.00,0.81}{#1}}
\newcommand{\BaseNTok}[1]{\textcolor[rgb]{0.00,0.00,0.81}{#1}}
\newcommand{\FloatTok}[1]{\textcolor[rgb]{0.00,0.00,0.81}{#1}}
\newcommand{\ConstantTok}[1]{\textcolor[rgb]{0.00,0.00,0.00}{#1}}
\newcommand{\CharTok}[1]{\textcolor[rgb]{0.31,0.60,0.02}{#1}}
\newcommand{\SpecialCharTok}[1]{\textcolor[rgb]{0.00,0.00,0.00}{#1}}
\newcommand{\StringTok}[1]{\textcolor[rgb]{0.31,0.60,0.02}{#1}}
\newcommand{\VerbatimStringTok}[1]{\textcolor[rgb]{0.31,0.60,0.02}{#1}}
\newcommand{\SpecialStringTok}[1]{\textcolor[rgb]{0.31,0.60,0.02}{#1}}
\newcommand{\ImportTok}[1]{#1}
\newcommand{\CommentTok}[1]{\textcolor[rgb]{0.56,0.35,0.01}{\textit{#1}}}
\newcommand{\DocumentationTok}[1]{\textcolor[rgb]{0.56,0.35,0.01}{\textbf{\textit{#1}}}}
\newcommand{\AnnotationTok}[1]{\textcolor[rgb]{0.56,0.35,0.01}{\textbf{\textit{#1}}}}
\newcommand{\CommentVarTok}[1]{\textcolor[rgb]{0.56,0.35,0.01}{\textbf{\textit{#1}}}}
\newcommand{\OtherTok}[1]{\textcolor[rgb]{0.56,0.35,0.01}{#1}}
\newcommand{\FunctionTok}[1]{\textcolor[rgb]{0.00,0.00,0.00}{#1}}
\newcommand{\VariableTok}[1]{\textcolor[rgb]{0.00,0.00,0.00}{#1}}
\newcommand{\ControlFlowTok}[1]{\textcolor[rgb]{0.13,0.29,0.53}{\textbf{#1}}}
\newcommand{\OperatorTok}[1]{\textcolor[rgb]{0.81,0.36,0.00}{\textbf{#1}}}
\newcommand{\BuiltInTok}[1]{#1}
\newcommand{\ExtensionTok}[1]{#1}
\newcommand{\PreprocessorTok}[1]{\textcolor[rgb]{0.56,0.35,0.01}{\textit{#1}}}
\newcommand{\AttributeTok}[1]{\textcolor[rgb]{0.77,0.63,0.00}{#1}}
\newcommand{\RegionMarkerTok}[1]{#1}
\newcommand{\InformationTok}[1]{\textcolor[rgb]{0.56,0.35,0.01}{\textbf{\textit{#1}}}}
\newcommand{\WarningTok}[1]{\textcolor[rgb]{0.56,0.35,0.01}{\textbf{\textit{#1}}}}
\newcommand{\AlertTok}[1]{\textcolor[rgb]{0.94,0.16,0.16}{#1}}
\newcommand{\ErrorTok}[1]{\textcolor[rgb]{0.64,0.00,0.00}{\textbf{#1}}}
\newcommand{\NormalTok}[1]{#1}
\usepackage{graphicx,grffile}
\makeatletter
\def\maxwidth{\ifdim\Gin@nat@width>\linewidth\linewidth\else\Gin@nat@width\fi}
\def\maxheight{\ifdim\Gin@nat@height>\textheight\textheight\else\Gin@nat@height\fi}
\makeatother
% Scale images if necessary, so that they will not overflow the page
% margins by default, and it is still possible to overwrite the defaults
% using explicit options in \includegraphics[width, height, ...]{}
\setkeys{Gin}{width=\maxwidth,height=\maxheight,keepaspectratio}
\IfFileExists{parskip.sty}{%
\usepackage{parskip}
}{% else
\setlength{\parindent}{0pt}
\setlength{\parskip}{6pt plus 2pt minus 1pt}
}
\setlength{\emergencystretch}{3em}  % prevent overfull lines
\providecommand{\tightlist}{%
  \setlength{\itemsep}{0pt}\setlength{\parskip}{0pt}}
\setcounter{secnumdepth}{0}
% Redefines (sub)paragraphs to behave more like sections
\ifx\paragraph\undefined\else
\let\oldparagraph\paragraph
\renewcommand{\paragraph}[1]{\oldparagraph{#1}\mbox{}}
\fi
\ifx\subparagraph\undefined\else
\let\oldsubparagraph\subparagraph
\renewcommand{\subparagraph}[1]{\oldsubparagraph{#1}\mbox{}}
\fi

%%% Use protect on footnotes to avoid problems with footnotes in titles
\let\rmarkdownfootnote\footnote%
\def\footnote{\protect\rmarkdownfootnote}

%%% Change title format to be more compact
\usepackage{titling}

% Create subtitle command for use in maketitle
\newcommand{\subtitle}[1]{
  \posttitle{
    \begin{center}\large#1\end{center}
    }
}

\setlength{\droptitle}{-2em}

  \title{FinalProjectRS JH}
    \pretitle{\vspace{\droptitle}\centering\huge}
  \posttitle{\par}
    \author{Alejandro Coy}
    \preauthor{\centering\large\emph}
  \postauthor{\par}
      \predate{\centering\large\emph}
  \postdate{\par}
    \date{2019-02-12}


\begin{document}
\maketitle

\subsubsection{Peer-graded Assignment: Course Project
2}\label{peer-graded-assignment-course-project-2}

\subsection{Introduction}\label{introduction}

Storms and other severe weather events can cause both public health and
economic problems for communities and municipalities. Many severe events
can result in fatalities, injuries, and property damage, and preventing
such outcomes to the extent possible is a key concern.

This project involves exploring the U.S. National Oceanic and
Atmospheric Administration's (NOAA) storm database. This database tracks
characteristics of major storms and weather events in the United States,
including when and where they occur, as well as estimates of any
fatalities, injuries, and property damage.

The two question this project addrest to respones are the following:

\begin{enumerate}
\def\labelenumi{\arabic{enumi}.}
\item
  Across the United States, which types of events are most harmful with
  respect to population health?
\item
  Across the United States, which types of events have the greatest
  economic consequences?
\end{enumerate}

\subsection{Data Analysis}\label{data-analysis}

The data was subset in two data frames aiming to response each question.
These two data frames contains just the columns necesseary for the
analysis. The data was fruther filter for rows with numeric values
greater than 0, grouped by group by event type and finally sumarized by
sum of the varible of interest.

\subsection{Importing Data}\label{importing-data}

The package readr is used to read the file and save it in a data frame
called ``data''. Summary of the data is presented using summary()
function.

\begin{Shaded}
\begin{Highlighting}[]
\KeywordTok{library}\NormalTok{(readr)}
\KeywordTok{library}\NormalTok{(dplyr)}
\end{Highlighting}
\end{Shaded}

\begin{verbatim}
## 
## Attaching package: 'dplyr'
\end{verbatim}

\begin{verbatim}
## The following objects are masked from 'package:stats':
## 
##     filter, lag
\end{verbatim}

\begin{verbatim}
## The following objects are masked from 'package:base':
## 
##     intersect, setdiff, setequal, union
\end{verbatim}

\begin{Shaded}
\begin{Highlighting}[]
\KeywordTok{library}\NormalTok{(ggplot2)}
\KeywordTok{library}\NormalTok{(tidyr)}

\NormalTok{data <-}\StringTok{ }\KeywordTok{read_csv}\NormalTok{(}\StringTok{"repdata_data_StormData.csv.bz2"}\NormalTok{)}
\end{Highlighting}
\end{Shaded}

\begin{verbatim}
## Parsed with column specification:
## cols(
##   .default = col_double(),
##   BGN_DATE = col_character(),
##   BGN_TIME = col_character(),
##   TIME_ZONE = col_character(),
##   COUNTYNAME = col_character(),
##   STATE = col_character(),
##   EVTYPE = col_character(),
##   BGN_AZI = col_logical(),
##   BGN_LOCATI = col_logical(),
##   END_DATE = col_logical(),
##   END_TIME = col_logical(),
##   COUNTYENDN = col_logical(),
##   END_AZI = col_logical(),
##   END_LOCATI = col_logical(),
##   PROPDMGEXP = col_character(),
##   CROPDMGEXP = col_logical(),
##   WFO = col_logical(),
##   STATEOFFIC = col_logical(),
##   ZONENAMES = col_logical(),
##   REMARKS = col_logical()
## )
\end{verbatim}

\begin{verbatim}
## See spec(...) for full column specifications.
\end{verbatim}

\begin{verbatim}
## Warning: 5255570 parsing failures.
##  row col           expected actual                             file
## 1671 WFO 1/0/T/F/TRUE/FALSE     NG 'repdata_data_StormData.csv.bz2'
## 1673 WFO 1/0/T/F/TRUE/FALSE     NG 'repdata_data_StormData.csv.bz2'
## 1674 WFO 1/0/T/F/TRUE/FALSE     NG 'repdata_data_StormData.csv.bz2'
## 1675 WFO 1/0/T/F/TRUE/FALSE     NG 'repdata_data_StormData.csv.bz2'
## 1678 WFO 1/0/T/F/TRUE/FALSE     NG 'repdata_data_StormData.csv.bz2'
## .... ... .................. ...... ................................
## See problems(...) for more details.
\end{verbatim}

\begin{Shaded}
\begin{Highlighting}[]
\KeywordTok{names}\NormalTok{(data)}
\end{Highlighting}
\end{Shaded}

\begin{verbatim}
##  [1] "STATE__"    "BGN_DATE"   "BGN_TIME"   "TIME_ZONE"  "COUNTY"    
##  [6] "COUNTYNAME" "STATE"      "EVTYPE"     "BGN_RANGE"  "BGN_AZI"   
## [11] "BGN_LOCATI" "END_DATE"   "END_TIME"   "COUNTY_END" "COUNTYENDN"
## [16] "END_RANGE"  "END_AZI"    "END_LOCATI" "LENGTH"     "WIDTH"     
## [21] "F"          "MAG"        "FATALITIES" "INJURIES"   "PROPDMG"   
## [26] "PROPDMGEXP" "CROPDMG"    "CROPDMGEXP" "WFO"        "STATEOFFIC"
## [31] "ZONENAMES"  "LATITUDE"   "LONGITUDE"  "LATITUDE_E" "LONGITUDE_"
## [36] "REMARKS"    "REFNUM"
\end{verbatim}

\begin{Shaded}
\begin{Highlighting}[]
\KeywordTok{summary}\NormalTok{(data)}
\end{Highlighting}
\end{Shaded}

\begin{verbatim}
##     STATE__       BGN_DATE           BGN_TIME          TIME_ZONE        
##  Min.   : 1.0   Length:902297      Length:902297      Length:902297     
##  1st Qu.:19.0   Class :character   Class :character   Class :character  
##  Median :30.0   Mode  :character   Mode  :character   Mode  :character  
##  Mean   :31.2                                                           
##  3rd Qu.:45.0                                                           
##  Max.   :95.0                                                           
##                                                                         
##      COUNTY       COUNTYNAME           STATE              EVTYPE         
##  Min.   :  0.0   Length:902297      Length:902297      Length:902297     
##  1st Qu.: 31.0   Class :character   Class :character   Class :character  
##  Median : 75.0   Mode  :character   Mode  :character   Mode  :character  
##  Mean   :100.6                                                           
##  3rd Qu.:131.0                                                           
##  Max.   :873.0                                                           
##                                                                          
##    BGN_RANGE        BGN_AZI        BGN_LOCATI     END_DATE      
##  Min.   :   0.000   Mode:logical   Mode:logical   Mode:logical  
##  1st Qu.:   0.000   NA's:902297    TRUE:1         NA's:902297   
##  Median :   0.000                  NA's:902296                  
##  Mean   :   1.484                                               
##  3rd Qu.:   1.000                                               
##  Max.   :3749.000                                               
##                                                                 
##  END_TIME         COUNTY_END COUNTYENDN       END_RANGE       
##  Mode:logical   Min.   :0    Mode:logical   Min.   :  0.0000  
##  NA's:902297    1st Qu.:0    NA's:902297    1st Qu.:  0.0000  
##                 Median :0                   Median :  0.0000  
##                 Mean   :0                   Mean   :  0.9862  
##                 3rd Qu.:0                   3rd Qu.:  0.0000  
##                 Max.   :0                   Max.   :925.0000  
##                                                               
##  END_AZI        END_LOCATI         LENGTH              WIDTH         
##  Mode:logical   Mode:logical   Min.   :   0.0000   Min.   :   0.000  
##  NA's:902297    NA's:902297    1st Qu.:   0.0000   1st Qu.:   0.000  
##                                Median :   0.0000   Median :   0.000  
##                                Mean   :   0.2301   Mean   :   7.503  
##                                3rd Qu.:   0.0000   3rd Qu.:   0.000  
##                                Max.   :2315.0000   Max.   :4400.000  
##                                                                      
##        F               MAG            FATALITIES          INJURIES        
##  Min.   :0.0      Min.   :    0.0   Min.   :  0.0000   Min.   :   0.0000  
##  1st Qu.:0.0      1st Qu.:    0.0   1st Qu.:  0.0000   1st Qu.:   0.0000  
##  Median :1.0      Median :   50.0   Median :  0.0000   Median :   0.0000  
##  Mean   :0.9      Mean   :   46.9   Mean   :  0.0168   Mean   :   0.1557  
##  3rd Qu.:1.0      3rd Qu.:   75.0   3rd Qu.:  0.0000   3rd Qu.:   0.0000  
##  Max.   :5.0      Max.   :22000.0   Max.   :583.0000   Max.   :1700.0000  
##  NA's   :843563                                                           
##     PROPDMG         PROPDMGEXP           CROPDMG        CROPDMGEXP     
##  Min.   :   0.00   Length:902297      Min.   :  0.000   Mode :logical  
##  1st Qu.:   0.00   Class :character   1st Qu.:  0.000   FALSE:19       
##  Median :   0.00   Mode  :character   Median :  0.000   NA's :902278   
##  Mean   :  12.06                      Mean   :  1.527                  
##  3rd Qu.:   0.50                      3rd Qu.:  0.000                  
##  Max.   :5000.00                      Max.   :990.000                  
##                                                                        
##    WFO          STATEOFFIC     ZONENAMES         LATITUDE   
##  Mode:logical   Mode:logical   Mode:logical   Min.   :   0  
##  TRUE:7166      NA's:902297    NA's:902297    1st Qu.:2802  
##  NA's:895131                                  Median :3540  
##                                               Mean   :2875  
##                                               3rd Qu.:4019  
##                                               Max.   :9706  
##                                               NA's   :47    
##    LONGITUDE        LATITUDE_E     LONGITUDE_     REMARKS       
##  Min.   :-14451   Min.   :   0   Min.   :-14455   Mode:logical  
##  1st Qu.:  7247   1st Qu.:   0   1st Qu.:     0   NA's:902297   
##  Median :  8707   Median :   0   Median :     0                 
##  Mean   :  6940   Mean   :1452   Mean   :  3509                 
##  3rd Qu.:  9605   3rd Qu.:3549   3rd Qu.:  8735                 
##  Max.   : 17124   Max.   :9706   Max.   :106220                 
##                   NA's   :40                                    
##      REFNUM      
##  Min.   :     1  
##  1st Qu.:225575  
##  Median :451149  
##  Mean   :451149  
##  3rd Qu.:676723  
##  Max.   :902297  
## 
\end{verbatim}

\subsection{Data Proccesing}\label{data-proccesing}

The data strategy was to divide the data in two dataframes to answer
each question.

\subsubsection{Most Harmful Events}\label{most-harmful-events}

In order to address the first question the amount of fatalities and
injures were considered. The strategy for this project was to add the
number of fatalities and injuries for each type of the event and then
rank them. The events with total incidents above 1000 were selected and
then the data was transformed to tidy data set using the gather function

All the analysis was done uing dplyr package.

\subsubsection{Most Costly Events}\label{most-costly-events}

To calculate the cost impact from the weather events the damge in
porperty and the damage in crops need to be considered. Due to how the
data is presented is needed it to tranlate the dollar amount reported
with the correpondent index to have all the cost in the same
units(billions of dollars).

\begin{Shaded}
\begin{Highlighting}[]
\NormalTok{valid_ex<-}\StringTok{ }\KeywordTok{c}\NormalTok{(}\StringTok{"M"}\NormalTok{,}\StringTok{"m"}\NormalTok{,}\StringTok{"B"}\NormalTok{,}\StringTok{"H"}\NormalTok{,}\StringTok{"h"}\NormalTok{,}\StringTok{"K"}\NormalTok{,}\StringTok{"k"}\NormalTok{)}
\NormalTok{pdamage <-}\StringTok{ }\NormalTok{data }\OperatorTok
\StringTok{  }\KeywordTok{select}\NormalTok{(EVTYPE,PROPDMG,PROPDMGEXP)}\OperatorTok
\StringTok{  }\KeywordTok{filter}\NormalTok{(}\OperatorTok{!}\KeywordTok{is.na}\NormalTok{(PROPDMGEXP))}\OperatorTok
\StringTok{  }\KeywordTok{group_by}\NormalTok{(PROPDMGEXP,EVTYPE)}\OperatorTok
\StringTok{  }\KeywordTok{summarise}\NormalTok{(}\DataTypeTok{total =} \KeywordTok{sum}\NormalTok{(PROPDMG))}\OperatorTok
\StringTok{  }\KeywordTok{filter}\NormalTok{(PROPDMGEXP }\OperatorTok\StringTok{ }\NormalTok{valid_ex)}

\NormalTok{fromMtoB <-}\StringTok{ }\NormalTok{pdamage}\OperatorTok{$}\NormalTok{PROPDMGEXP }\OperatorTok{==}\StringTok{ "M"} \OperatorTok{|}\StringTok{ }\NormalTok{pdamage}\OperatorTok{$}\NormalTok{PROPDMGEXP }\OperatorTok{==}\StringTok{ "m"}
\NormalTok{fromktoB <-}\StringTok{ }\NormalTok{pdamage}\OperatorTok{$}\NormalTok{PROPDMGEXP }\OperatorTok{==}\StringTok{ "K"} \OperatorTok{|}\StringTok{ }\NormalTok{pdamage}\OperatorTok{$}\NormalTok{PROPDMGEXP }\OperatorTok{==}\StringTok{ "k"}
\NormalTok{fromhtoB <-}\StringTok{ }\NormalTok{pdamage}\OperatorTok{$}\NormalTok{PROPDMGEXP }\OperatorTok{==}\StringTok{ "H"} \OperatorTok{|}\StringTok{ }\NormalTok{pdamage}\OperatorTok{$}\NormalTok{PROPDMGEXP }\OperatorTok{==}\StringTok{ "h"}

\NormalTok{pdamage}\OperatorTok{$}\NormalTok{total[fromMtoB] <-}\StringTok{ }\NormalTok{pdamage}\OperatorTok{$}\NormalTok{total[fromMtoB]}\OperatorTok{/}\DecValTok{1000}
\NormalTok{pdamage}\OperatorTok{$}\NormalTok{total[fromktoB] <-}\StringTok{ }\NormalTok{pdamage}\OperatorTok{$}\NormalTok{total[fromktoB]}\OperatorTok{/}\DecValTok{1000000}
\NormalTok{pdamage}\OperatorTok{$}\NormalTok{total[fromhtoB] <-}\StringTok{ }\NormalTok{pdamage}\OperatorTok{$}\NormalTok{total[fromhtoB]}\OperatorTok{/}\DecValTok{1000000000}
\end{Highlighting}
\end{Shaded}

Once the dollar amount is transfromed in the same units, the total
amount can be calculated and the events are rank in descendet order.

\begin{Shaded}
\begin{Highlighting}[]
\NormalTok{pdamageTotal <-}\StringTok{ }\NormalTok{pdamage }\OperatorTok
\StringTok{  }\KeywordTok{group_by}\NormalTok{(EVTYPE) }\OperatorTok
\StringTok{  }\KeywordTok{summarize}\NormalTok{(}\DataTypeTok{totalProp_damage =} \KeywordTok{sum}\NormalTok{(total))}\OperatorTok
\StringTok{  }\KeywordTok{arrange}\NormalTok{(}\KeywordTok{desc}\NormalTok{(totalProp_damage))}
\end{Highlighting}
\end{Shaded}

For the crop damage there were not index reports, so it is assumed that
the amount presented is in dollars. The data is grouped by event and the
total amount in damage to the crops is summarized.

\begin{Shaded}
\begin{Highlighting}[]
\CommentTok{#cropDamage}
\NormalTok{cdamage <-}\StringTok{ }\NormalTok{data }\OperatorTok
\StringTok{  }\KeywordTok{select}\NormalTok{(EVTYPE,CROPDMG,CROPDMGEXP)}\OperatorTok
\StringTok{  }\KeywordTok{filter}\NormalTok{(CROPDMG }\OperatorTok{>}\DecValTok{0}\NormalTok{)}\OperatorTok
\StringTok{  }\KeywordTok{group_by}\NormalTok{(EVTYPE)}\OperatorTok
\StringTok{  }\KeywordTok{summarise}\NormalTok{(}\DataTypeTok{totalCrop_damage =} \KeywordTok{sum}\NormalTok{(CROPDMG)}\OperatorTok{/}\DecValTok{1000000}\NormalTok{)}\OperatorTok
\StringTok{  }\KeywordTok{arrange}\NormalTok{(}\KeywordTok{desc}\NormalTok{(totalCrop_damage))}
\end{Highlighting}
\end{Shaded}

Finally the data for property damage and crop damage is joined in the
compilation data frame and a new column is created with the sum of
property damage and crop damage for each event type.

\begin{Shaded}
\begin{Highlighting}[]
\NormalTok{compilation <-}\StringTok{ }\KeywordTok{left_join}\NormalTok{(pdamageTotal,cdamage,}\DataTypeTok{by=}\StringTok{"EVTYPE"}\NormalTok{) }\OperatorTok
\StringTok{  }\CommentTok{#replace Na with 0}
\StringTok{  }\KeywordTok{mutate_all}\NormalTok{(}\KeywordTok{funs}\NormalTok{(}\KeywordTok{replace}\NormalTok{(.,}\KeywordTok{is.na}\NormalTok{(.),}\DecValTok{0}\NormalTok{))) }\OperatorTok
\StringTok{  }\KeywordTok{mutate}\NormalTok{(}\DataTypeTok{Total =}\NormalTok{ totalCrop_damage }\OperatorTok{+}\StringTok{ }\NormalTok{totalProp_damage)}
\CommentTok{#Events filter with more than 2 billion in damage}
\end{Highlighting}
\end{Shaded}

\subsection{Results}\label{results}

\subsubsection{Most Harmful Weather Events ins
USA(1950-2011)}\label{most-harmful-weather-events-ins-usa1950-2011}

To response the first questions the type of plot selected was a bar plot
with the coordinates flip so the events are in the Y axis. As can been
seen from the graph is clearly that Tornado is the most Harmful event in
the USA in the period of 1950-2011.

\begin{Shaded}
\begin{Highlighting}[]
\KeywordTok{ggplot}\NormalTok{(by_event)}\OperatorTok{+}
\StringTok{  }\KeywordTok{aes}\NormalTok{(}\DataTypeTok{x=}\KeywordTok{reorder}\NormalTok{(EVTYPE,}\OperatorTok{-}\NormalTok{TOTAL),}\DataTypeTok{y=}\NormalTok{ Amount, }\DataTypeTok{fill=}\NormalTok{ Harm)}\OperatorTok{+}
\StringTok{  }\KeywordTok{geom_bar}\NormalTok{(}\DataTypeTok{stat =} \StringTok{"identity"}\NormalTok{)}\OperatorTok{+}
\StringTok{  }\KeywordTok{coord_flip}\NormalTok{()}\OperatorTok{+}
\StringTok{  }\KeywordTok{labs}\NormalTok{(}\DataTypeTok{y=}\StringTok{"Total Harmful Events"}\NormalTok{, }\DataTypeTok{x =} \StringTok{""}\NormalTok{,}\DataTypeTok{title=}\StringTok{"Weather Events Most Harmful to Population USA(1950-2011)"}\NormalTok{)}\OperatorTok{+}
\StringTok{  }\KeywordTok{theme_minimal}\NormalTok{()}
\end{Highlighting}
\end{Shaded}

\includegraphics{FinalProjecRS_JH_files/figure-latex/unnamed-chunk-7-1.pdf}

\subsubsection{Most Costly Weather Events ins
USA(1950-2011)}\label{most-costly-weather-events-ins-usa1950-2011}

The events with cost related damege over 2 billions are ploted. In terms
of the cost the flood event is ranked number one follow by Hurricane and
Tornados. Once again ggplot is used to plot the results.

\begin{Shaded}
\begin{Highlighting}[]
\KeywordTok{ggplot}\NormalTok{(}\KeywordTok{filter}\NormalTok{(compilation,compilation}\OperatorTok{$}\NormalTok{Total }\OperatorTok{>}\StringTok{ }\DecValTok{2}\NormalTok{))}\OperatorTok{+}
\StringTok{  }\KeywordTok{aes}\NormalTok{(}\DataTypeTok{x=}\KeywordTok{reorder}\NormalTok{(EVTYPE,}\OperatorTok{-}\NormalTok{Total),}\DataTypeTok{y=}\NormalTok{ Total)}\OperatorTok{+}
\StringTok{  }\KeywordTok{geom_bar}\NormalTok{(}\DataTypeTok{stat =} \StringTok{"identity"}\NormalTok{,}\DataTypeTok{fill=}\StringTok{"brown"}\NormalTok{)}\OperatorTok{+}
\StringTok{  }\KeywordTok{coord_flip}\NormalTok{()}\OperatorTok{+}
\StringTok{  }\KeywordTok{labs}\NormalTok{(}\DataTypeTok{y=}\StringTok{"Total Cost in damage(Billions)"}\NormalTok{, }\DataTypeTok{x =} \StringTok{""}\NormalTok{,}\DataTypeTok{title=}\StringTok{"Weather Events Most Costly in USA(1950-2011)"}\NormalTok{)}\OperatorTok{+}
\StringTok{  }\KeywordTok{theme_minimal}\NormalTok{()}
\end{Highlighting}
\end{Shaded}

\includegraphics{FinalProjecRS_JH_files/figure-latex/unnamed-chunk-8-1.pdf}


\end{document}
